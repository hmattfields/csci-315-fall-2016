\documentclass[12pt]{article}
\usepackage{listings}
\usepackage{color}
\textwidth=7in
\textheight=9.5in
\topmargin=-1in
\headheight=0in
\headsep=.5in
\hoffset  -.85in

\definecolor{mygray}{rgb}{0.4,0.4,0.4}
\definecolor{mygreen}{rgb}{0,0.8,0.6}
\definecolor{myorange}{rgb}{1.0,0.4,0}

\lstset{
basicstyle=\footnotesize\sffamily\color{black},
commentstyle=\color{mygray},
frame=single,
numbers=left,
numbersep=5pt,
numberstyle=\tiny\color{mygray},
keywordstyle=\color{mygreen},
showspaces=false,
showstringspaces=false,
stringstyle=\color{myorange},
tabsize=2
}

\pagestyle{empty}

\renewcommand{\thefootnote}{\fnsymbol{footnote}}

\begin{document}

\begin{center}
{\bf Lab 0: Using Git and the Command Line
}
\end{center}

\setlength{\unitlength}{1in}

\begin{picture}(6,.1) 
\put(0,0) {\line(1,0){6.25}}         
\end{picture}

\renewcommand{\arraystretch}{2}
\setlength{\tabcolsep}{6pt} % General space between cols (6pt standard)
\renewcommand{\arraystretch}{.5} % General space between rows (1 standard)

\vskip.15in
\noindent\textbf{Instructions:} In this lab we will get use to using the Git (via GitHub) with a Command Line Interface (CLI) environment.  You will create a GitHub account, fork this classes repository, do some basic bug fixing, and then submit via GitHub.
\begin{enumerate}
\item Visit http://www.github.com and create an account if you do not already have one.
\item Email me your account name.  My email is pwest@csuniv.edu.
\item Once I email you back you should be part of the CSU-CS organization and have access to the csci-315-fall-2105 repository.
\item Fork the repository.
\item On you local machine checkout the repository: \$ git clone $<$url to your repository$>$
\item Fix the bugs in lab00 (see below for more instructions.)
\item Run some tests to confirm everything is passing.
\item Mark your changes for commit: \$ git add helloworld.cpp main.cpp
\item Commit your changes: \$ git commit
\item Push your changes so I have access to them: \$ git push
\end{enumerate}

This lab00 requires a function in helloworld.cpp:
\begin{lstlisting}[language=C++]{Name=test2}
/* This function shall return the string "Hello World!" */
string getHelloMessage();
\end{lstlisting}

In addition, write a main.cpp file that takes an integer $i$ as a parameter and prints the return of getHelloMessage() $i$ times.

\vskip.15in
\noindent\textbf{Example Input 1:} \\
\$ ./helloworld 2

\vskip.15in
\noindent\textbf{Example Output 1:} \\
Hello World! \\
Hello World!

\vskip.15in
\noindent\textbf{Example Input 2:} \\
\$ ./helloworld -1

\vskip.15in
\noindent\textbf{Example Output 2:}

\vskip.15in
\noindent\textbf{Example Input 3:} \\
\$ ./helloworld 1

\vskip.15in
\noindent\textbf{Example Output 3:} \\
Hello World!

\vskip.15in
\noindent\textbf{How to test:} \\
Some tests have been written, do a \$ make test to see the results of testing.

\vskip.15in
\noindent\textbf{How to turn in:} \\
Turn in via GitHub.  Ensure the file(s) are in your lab00 directory and then:
\begin{itemize}
\item \$ git add $<$files$>$
\item \$ git commit 
\item \$ git push
\end{itemize}

\vskip.15in
\noindent\textbf{Due Date:}
August 26, 2015 2359

\vskip.15in
\noindent\textbf{Teamwork:} No teamwork, your work must be your own.

\end{document}
